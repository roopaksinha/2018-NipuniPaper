\section{Conclusions and Future Works}

We propose an Automatic Translation Tool that takes in UML Activity Diagrams and generates Java code for smart home apps. Starting with a systematic literature review, we identified the useful features of a usable app design model. We also found that current design models do not completely support these factors. Following a qualitative review of UML diagrams, we identify UML Activity Diagrams as the best available option for modelling smart home apps. We then focussed on architecting a solution for generating executable code from UML Activity Diagrams. We identified primary architectural drivers, which included usability and modularity as the prime quality attributes. We then presented both the logical and process views of this architecture, built using architectural patterns and tactics that supported the architectural drivers. Then, we developed the automatic translation tool which carries out a number of model transformations to convert visual designs in Java code. These visual designs are created in a model editor that we also developed. We evaluated the tool through four representative case studies, and observed that it successfully achieves the primary architectural drivers, albeit it does need some further development to completely automate the compilation process. 

Other future directions include adapting UML Activity Diagrams for specific domains such as primary health such that domain experts can build app models using vocabulary that they are more familiar with. 
In terms of the Java code generation, code is generated for the respective activities modeled in the Activity Diagram Model Editor, therefore code generated from the tool only defines the behavior of an object and does not include the structure of the code such as class and method declarations. Currently, each design is contained within its own class. For larger problems, the tool should allow the user to provide some structural information as well. 
Hence, a more specific modeling language  that supports both behavioral and structural aspects of a system could be chosen. 
